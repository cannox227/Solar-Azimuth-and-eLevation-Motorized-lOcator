\chapter{Conclusioni}

\hypertarget{resoconto-esperienza}{%
\section{\texorpdfstring{Resoconto esperienza
}{Resoconto esperienza }}\label{resoconto-esperienza}}

Il corso è stato formativo sotto tutti i punti di vista poiché abbiamo
potuto realizzare una scheda partendo da zero, cosa che negli altri
corsi non avevamo mai fatto. Per alcuni quindi è stata la prima PCB
progettata, perciò una tappa molto importante per il loro lavoro o
studio futuro. Inoltre il fatto di aver programmato il microcontrollore
con le poche basi di programmazione che avevamo dai corsi precedenti ci
ha permesso di imparare molto a livello firmware. Siccome il progetto
non è ancora completo, l'intenzione è quella di aggiungere man mano i
componenti mancanti al fine di avere la scheda più simile possibile a
quanto era stato programmato. Per concludere, i concetti imparati nel
modulo 1 del corso sono stati di fondamentale importanza per la
realizzazione della scheda, senza i quali sarebbe stato molto difficile
avere un'idea di come impostare il lavoro. L'obiettivo di progettare in
modo consono una scheda elettronica è dunque stato raggiunto.

\hypertarget{lezioni-imparate}{%
\section{Lezioni imparate}\label{Lezioni-imparate}}

\noindent Ragionando a posteriori su quanto abbiamo fatto, abbiamo stilato un
elenco di cosa avremmo potuto fare meglio e che cercheremo di applicare
nei progetti futuri:

\begin{itemize}
\item
  \begin{quote}
  Il fanout del microcontrollore lo avremmo potuto fare esclusivamente
  con tracce da 0.2mm, senza ricorrere ad espansioni a 0.3mm, almeno per
  le tracce più corte. Questo avrebbe semplificato di non poco il
  routing di quella sezione;
  \end{quote}
\item
  \begin{quote}
  Considerare fin da subito i return paths per i condensatori di
  decoupling. Nella realizzazione non abbiamo pensato a collegare alcuna
  traccia per il GND, perché ci siamo affidati ai poligoni di
  riempimento. Questo si è rivelato un problema, perché alla fine del
  progetto abbiamo riscontrato che molti riempimenti risultavano
  scollegati da GND e, soprattutto, che i percorsi effettivi delle linee
  di GND dei condensatori di bypass del microcontrollore sono risultati
  molto lunghi, il che rende quasi inutile il fatto di posizionarli
  vicini ai pin;
  \end{quote}
\item
  \begin{quote}
  Controllare meglio il posizionamento dei refdes sul layer di
  silkscreen. Abbiamo realizzato solo a scheda prodotta che il refdes di
  C10 è stato posizionato sopra C14, rendendolo invisibile;
  \end{quote}
\item
  \begin{quote}
  Controllare meglio i componenti scelti. Abbiamo infatti scelto di
  utilizzare un led rgb di un formato che in realtà si fatica a trovare sul
  mercato. Anche questo è stato scoperto solo in fase di assemblaggio.
  \end{quote}
\end{itemize}
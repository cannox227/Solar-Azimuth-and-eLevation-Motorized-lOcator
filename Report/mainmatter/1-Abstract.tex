\chapter{Abstract}

La seconda parte del corso di Progettazione e Prototipazione di Sistemi
Elettronici prevede la completa progettazione di una scheda elettronica
che, nel nostro caso, ha come scopo quello di orientare in maniera
ottimale un pannello fotovoltaico verso la posizione del sole nel
cielo.\\
Per rendere più funzionale possibile ``\emph{SALMO}'' si è deciso di
rendere il tutto indipendente dalla posizione geografica, dalla
collocazione spaziale del pannello fotovoltaico ed ovviamente
dall'orario e dal giorno dell'anno. Il movimento del pannello è stato
progettato in modo che la sua posizione sia sempre il più possibile
perpendicolare ai raggi solari affinché la potenza generata dal pannello
possa essere sempre massima.\\
Per raggiungere l'obiettivo sono state stilate le seguenti specifiche:
due motori passo-passo unipolari per il movimento sui due assi (Azimuth
e Elevation), \emph{GPS} per rilevare la posizione geografica,
magnetometro ed accelerometro per il feedback della posizione del
pannello, circuiti di misura per tensione e corrente del pannello,
display oled per la visualizzazione dei dati mediante interfaccia utente
ed ovviamente un pannello fotovoltaico.

La progettazione della scheda è stata svolta in maniera precisa,
rigorosa ed organizzata, seguendo quindi dei passi ben definiti in modo
che ogni membro del team potesse lavorare parallelamente agli altri ed
allo stesso tempo contribuire al lavoro di tutti senza perdere step
fondamentali.\\
Inizialmente, facendo uso dei datasheet, abbiamo studiato
approfonditamente le caratteristiche del microcontrollore \emph{RP2040}
e dei componenti principali della scheda, al fine di poter progettare la
\emph{PCB} in maniera più consapevole. Ogni studente del corso ha poi
esposto alcune caratteristiche del microcontrollore, a partire dalla
memoria fino alla struttura fisica passando per le varie interfacce di
comunicazione, \emph{l'ADC}, i \emph{GPIO} e l'alimentazione.\\
Ogni gruppo ha poi concordato con il professore il progetto da
realizzare, fissando delle scadenze per la consegna dei vari schematici
ed infine per la \emph{BOM}.\\
Sono state decise inoltre delle specifiche di progetto uguali per ogni
gruppo (come ad esempio le dimensioni della scheda (10x6 cm), il
regolatore di tensione ecc..) in modo da avere delle BOM più uniformi
possibili, senza decine di componenti alternativi non necessari. Dopo
l'avvio vero e proprio del progetto in autonomia, le lezioni svolte in
classe o in laboratorio erano principalmente volte alla correzione, alla
revisione ed al debugging, facendo sì che ogni dubbio potesse essere
chiarito immediatamente.\\
Finito lo sbroglio circuitale siamo passati alla generazione e all'invio
dei file gerber al Professore, che ha successivamente provveduto a
spedire questi ultimi al produttore. Analogamente, dopo aver completato
la \emph{BOM}, questa è stata inviata al Professore per il conseguente
acquisto del materiale necessario. In seguito all'arrivo delle
\emph{PCB} e dei componenti essenziali, abbiamo iniziato l'assemblaggio
sfruttando la strumentazione offerta dal laboratorio di elettronica
presente all'interno del FabLab. Dopo aver completato l'assemblaggio,
realizzato gran parte del firmware, svolto test a freddo prima (senza
alimentazione) ed a caldo successivamente (con +5V USB e +12V), possiamo
affermare che, benché ci siano ancora dei componenti mancanti, il
prototipo è completamente funzionante (a meno di un piccolo errore, vedi
cap. \protect\hyperlink{led-rgb}{\underline{LED RGB}} ) .

Tutto il progetto della \emph{PCB} è stato svolto utilizzando il
software open-source KiCad 6.0, mentre per la progettazione del firmware
abbiamo utilizzato l'\emph{IDE VS Code} e la compilazione manuale via
\emph{CMake}.\\
Per la gestione dell'intero progetto abbiamo deciso di avvalerci di un
\emph{VCS (Version Control System)}: git.\\
La repository è hostata su GitHub e può essere visionata cliccando sul
seguente link:\\
\href{https://github.com/thomasnonis/ppse-2021}{\underline{https://github.com/thomasnonis/ppse-2021}}.
